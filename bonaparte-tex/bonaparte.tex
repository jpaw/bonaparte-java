\documentclass[11pt,a4paper,oneside]{article}
% \usepackage{ngerman}
\usepackage{ucs}
\usepackage{a4wide}
\usepackage[utf8x]{inputenc}
\usepackage{listings}
    \lstset{numbers=left, numberstyle=\tiny, numbersep=5pt,frame=leftline}
    \lstset{language=Java} 
    \lstset{basicstyle=\small,tabsize=4,extendedchars=false,columns=flexible}
%    \lstset{keywordstyle=\ttfamily\bfseries}
    \lstset{keywordstyle=\bfseries}
    \lstset{identifierstyle=\ttfamily}
    \lstset{stringstyle=\rmfamily,showstringspaces=false}
    \lstset{commentstyle=\rmfamily\itshape}
\usepackage{graphicx}
\usepackage{geometry}
\geometry{a4paper,left=20mm,right=20mm, top=1cm, bottom=2cm, includeheadfoot}
\usepackage[pdfborder={0 0 0}]{hyperref}

\usepackage{rail}
% \hypersetup{pdfborder=0 0 0}  evtl auch 4 Nullen angeben!

%\title{Atbon - Almost Text Based Object Notation}
\title{BONAPARTE - Binary Object Notation As Portable And Readable Text
Extension\footnote{SHIPANWABE - Should Have I Picked Another Name With A Better Expansion?}
}
\author{Michael Bischoff}
\begin{document}
\maketitle
\begin{abstract}
Bonaparte is an Eclipse XText based domain specific language (DSL), with the purpose to generate code for fast object serialization / deserialization.
There is a preferred serialization format (the ``Bonaparte format''), but the DSL itself is independent of the serialization format. 
The Bonaparte serialization format combines features of CSV, JSON, Java serialization and Google protocol buffers.
\end{abstract}
\section{Why another serialization format?}
Object serialization is an important function in today's distributed computing
environments. No matter if you want to exchange synchronous or asynchronous
messages in a service-oriented modular software environment, persist
arbitrary data in some NoSQL database, use distributed caches, or just dump data
to a file.
Because it's used in so many places, often transparent to the application,
serialization / deserialization (or marshalling /unmarshalling) must be fast. In
addition, a compact presentation is important, because it helps reducing
required storage space and network traffic.
Many implementations sacrifice readability or portability for size. Others (like XML)
emphasize readability but neglect the message size.

The Bonaparte message format was designed with the following goals in mind:
\begin{itemize}
  \item Small message size without sacrifizing portability (computer byte order
  / programming language independence)
  \item fast serialization / deserialization
  \item good support for typical data types found in nowadays programming
  languages
  \item support of object inheritance
  \item extensibility of the message format
  \item compatibility of serialized formats if the underlying objects are
  changed
\end{itemize}
Normally, Bonaparte serialized formats are defined as UTF-8 encoded character byte streams.
Using an extension mechanism or via external setting of the encoding, other encodings are also supported (provided they cover all required code points of transmitted text fields).
In fact, the serialized format can be converted between UTF-8 and other formats during transmission and parsed in a different encoding. 

\section{Why a specialized DSL?}
The key driver for the development of the DSL was the desire to have a notation which is independent of the target language and also very compact.
The advantage of a DSL (and especially DSLs based on Eclipse XText) is, that you can get full syntax checking and syntax highlighting while entering the code
(this of course has the prerequisite that you use the Eclipse IDE). In addition, DSLs offer the most compact notation for a given purpose. If you don't like boilerplate code, DSLs are for you!

The grammar elements are
\begin{itemize}
  \item packages / modules
  \item type definitions (C / C++ developers miss those in Java)
  \item classes / objects (including enums)
  \item data fields / member variables
\end{itemize}
The grammar of the DSL is loosely derived from the Java language syntax.

At the moment, the only open source code generator is for the Java programming language. A generator for Google DART is planned as soon as dynamic class loading is supported by DART.

Generic key properties are defined in the core DSL. The DSL can be extended to
provide additional plausibility checks / restrictions, additional language components
(keywords, annotations) or output formats. Thus, the core DSL can be kept clean
of aspects used for exotic languages only, for example restrictions on field
name lengths (SQL, COBOL), or annotations specifying alternate names (as in
JPA).

In addition to the code generator, for every supported language a basic library providing one or more
implementations of support functions for the marshaller / unmarshaller are provided.

Lastly, adapters to use the functionality in commonly used libraries are
provided.


\section{Version compatibility}
Software systems are a volatile world. New features require extensions of the
data model on a regular base. In distributed environments, new releases
cannot be deployed on all nodes at the same time. This means that some nodes run
the older software release and must cooperate with nodes using the new software
release already. Even in single node environments, where objects have been
persisted using an older release, and will be read back into memory with the new
software release, we require interoperability of different versions.
Therefore, version compatibility is one of the core concepts of Bonaparte.

Bonaparte distinguishes two kinds of object changes.

\subsection{Compatible changes}
A compatible change is defined as a change of the object / class definition
which allows straightforward deserialization of an object serialized with the
older version. For the Bonaparte format, the following object changes are
compatible:
\begin{itemize}
  \item Extension of the maximum length of a field. Every field is stored
  using its actual size only, not its maximum size. Therefore, length increases
  do not cause any issues.
  \item Adding optional (nullable) fields at the end of an object. If an
  ``end of object'' code is encountered during the parse process, while an
  additional object is expected, the object is interpreted as if it had been
  stred with a null value.
  \item Certain type conversions. Conversions from unsigned to signed numbers,
  conversions from ascii to unicode types, or conversion of a boolean to a
  numeric type or from any numeric data type to an ascii or unicode type is
  possible. This is possible, because alphanumeric types are not enclosed in
  quotes in the serialized format.
  \item Changing a required field to optional / nullable.
  \item Renaming a field (unless meta data is used, but in that case, it is
  typically used to dynamically create the receiving objects anyway, which
  implies that both sides will be in sync).
\end{itemize}
% TODO: provide diagram outlining all conversions
The reverse will usually cause parsing errors. It is planned as a future
extension to allow configuration of the parser to accept the following reverse
changes:
\begin{itemize}
  \item Reduction of maximum length of alphanumeric fields (implemented by
  truncation of data where the data length exceeds the maximum allowed field
  length).
  \item Ignoring extra fields at the end of an object. (Null fields at the end
  of an object will be ignored by default)
\end{itemize}

\subsection{Incompatible changes}
Sometimes, an incompatible change cannot be avoided (adding required fields,
cleaning up by removing no longer used fields). A parser encountering an
incompatible change will break. Therefore, the concept of object revisions
(version numbers) is part of the grammar definition. The revision can be seen as
a second part of an object's class name. A later release will add functionality
to provide parsers for all current object revisions, and allow manually written
converters to be hooked into, such that all object revisions will be converted
to the latest revision.





\section{The Bonaparte serialization format in detail}
The Bonaparte serialization format supports the following specific features. We believe that no other
currently existing format supports all of them.
\begin{itemize}
    \item Smart escaping: Instead of just prepending escape characters to
    characters of special meaning (with the effect that some occurences will be escape characters, some others not, like in CSV (,) or SQL (')), escape characters or other special
    characters will always have their special meaning, while payload characters of this value will be represented by tokens
    (such as in markup languages). In this case, due to the limited number of special characters, the token will be a single
    character.
    \item Object oriented features: Whenever an instance of a specific class is
    serialized, the grammar can specify if exactly that class, or also any
    subclass can be (de-)serialized instead.
    \item Data domains. The implemention can provide a mapping of HQONs (Half
    Qualified Object Names) to class names. 
    \item Auto-detection of encoding conversions (for certain formats: one single-byte, one multi-byte) $=>$ use the EURO currency
    sign as criteria:
    ISO 8859-15, UTF-8? (Future feature, requires extension parsing)
    \item DSL to define message format without boilerplate code
    \item Object versioning for incompatible changes (a high level concept exists at the moment, implementation is planned as a future extension)
    \item Smart metadata handling. Some formats do not store metadata at all,
    some accompany parts of it with every record (field names for JSON / XML).
    Bonaparte support configurable metadata sending at transmission begin, or
    lazyly, once an object is required. The metadata contains enough information
    to allow the dynamic creation of the data classes and is sent as data
    objects defined in the ``meta'' package.
    \item Generics support. At the moment, it's the same halfhearted approach as in Java (namely, with type erasure), but the plan is to extend it to
    a full-featured implementation, comparable to C\# or C++'s templates. 
\end{itemize}

\section{Description of the format}
A serialized object in Bonaparte fromat is a series of bytes, where text fields
(strings) are represented by their UTF-8 encoded presentation, unless they
contain control characters, which are the only characters requiring escaping.

Bonaparte can directly support the following elementary data types:
\begin{itemize}
\item booleans. A boolean is either true (represented by the digit 1) or false (represented by the digit 0).
\item numbers. A number is of the pattern [-]{0-9}*[.]{0-9}*[e[-]{0-9}+].
The Bonaparte DSL allows to distinguish between int, long, float, double, or a
specification of an integral type with a maximum number of digits.
\item strings. (Either ASCII or Unicode.) A string is a sequence of characters,
in message format corresponding notation. Control characters (characters with
codepoint < 0x20) are escaped as follows: Escape character (ctrl-C), followed by
a character with codepoint 0x40 to 0x5f. The resulting character will be the one
with the codepoint of the second byte, modulus 32. Often such control characters
are not permitted due to business reasons, in which case the string serialized
form does not rely on escaping at all. A special control character is directly
converted into a space (ctrl-E), with the intent that parsers stripping leading
and trailing spaces treat this as a non-strippable space character.
\item Timestamp. A timestamp is represented in the following form:
YYYYMMDDHHMISS(fff), or any subsequence of this which at least defines the
day. The timestamp is assumed to be in UTC, Gregorian Calendar. The maximum
number of sub-second fractional digits is implementation defined,
implementations which see more digits presented than they support are expected to silently ignore them.  
\item raw data (binary bytes). Binary data is transmitted in base64 encoding.
\end{itemize}
Any elementary field is terminated by a field terminator character (ctrl-F). Any
field may be empty, which means, it consists only of the field terminator
character. If during parsing of an object the object termination character
(ctrl-O) is found, any additional fields are assumed to be empty. 

Arrays are represented by an introduction character ctrl-B, followed by an
arbitrary number of objects or elementary data items, followed by the array
termination character, ctrl-A. Converting a singular data item in an object into
an array therefore is (from a message presentation form) an upwards compatible operation. 

Objects are represented by an identifier giving their name (any valid Java
identifier of up to 30 characters length is fine) and a version number (default
0). Both object ID and version number are treated as regular fields, i.e. are
terminated by a field termination character. 

\subsection{Special characters}

\begin{tabular}{|c|c|c|p{8cm}|}
\hline
Hex & Char & Meaning & Comments \\
\hline
00 & \textasciicircum {@} & N/A & Does not appear, because it is used as a
string terminator character in some languages \\
01 & \textasciicircum A & array terminator  & is inserted after the last array
entry. Is not followed by an additional field terminator \\
02 & \textasciicircum B & array start       & Immediately followed by the number
of entries (0 .. n), then field terminator, then the entries itself. Used for array, List and Set data types. \\
03 & \textasciicircum C & comment record    & everything until the next
record terminator is a comment. Escaping of special characters does apply within
comments.
\\
04 & \textasciicircum D & unused            & allows creation of input files
via Unix console \\
05 & \textasciicircum E & escape char       & allows the transmission of control
characters in String fields. Is followed by the control character, offset by 0x40 ({@}) \\
06 & \textasciicircum F & field terminator  & is sent after every atomic field
\\
07 & \textasciicircum G &       -            & \\
08 & \textasciicircum H & unused             & reserved for backspace while
typing \\
09 & \textasciicircum I &       tab          & if escape characters are
allowed, a tab is represented via the normal tab character inside unicode
fields \\
0A & \textasciicircum J & record terminator & terminates a record (in files,
matches the newline character) \\
0B & \textasciicircum K &       -            & \\
0C & \textasciicircum L &       -            & \\
0D & \textasciicircum M & ignored           & is used immediately ahead of the record
terminator character in some operating systems, will cause an error if not
immediately followed by \textasciicircum J \\
0E & \textasciicircum N & NULL & used to serialise a field with null contents.
This token is not followed by a field terminator. Allows distinguishing between
an empty string (zero characters) and a null string.\\
0F & \textasciicircum O &
object terminator & sent after the last field of an object has been serialized\\
10 & \textasciicircum P & parent separator   & separates fields of parent object
from next derived object in order to allow extensions of objects at every level.\\
11 & \textasciicircum Q & quit      & close the connection (optional, after
transmission end)
\\
12 & \textasciicircum R & record start       & starts every record, is
followed by a version number (or blank) and a field terminator, to indicate
the format version (currently always 0 / blank) \\
\hline
\end{tabular}

\begin{tabular}{|c|c|c|p{8cm}|}
\hline
Hex & Char & Meaning & Comments \\
\hline
13 & \textasciicircum S & start object       & start a new (sub-) object. Is
followed by the object name (with optional package prefix in Java-style dot notation (minus a
configurable start package prefix)), then a field terminator, then the object's
version number (empty if default / 0), and another field terminator, then the
object's fields \\
14 & \textasciicircum T & transmission start & starts a multi-record block, is
followed by a version number (or blank) and a field terminator, to indicate
the format version (currently always 0 / blank) \\
15 & \textasciicircum U & transmission end   & follows the last record of a
transmission. Optionally followed by a ctrl-Z character \\
16 & \textasciicircum V &       -            & \\
17 & \textasciicircum W &       -            & \\
18 & \textasciicircum X & extension start   & provides optional extension information,
usually sent once at the start of a transmission. Is followed by one character
defining the type of the extension, then extension specific data. Currently,
only E is defined (charset encoding), followed by a Euro sign, which defines
either UTF-8 or ISO-8859-15 or we1252-? (standard character set converters will
convert this character as well, allowing an auto-detection of the char set)\\
19 & \textasciicircum Y & end of extension  & is sent after extension information ends.
Allows parsers to skip all (unknown) extensions. \\
1A & \textasciicircum Z & - & optionally send after the end of data
transmission character. closes connection (for example for sockets). \\
1B & \textasciicircum [ &       -            & \\
1C & \textasciicircum \textbackslash &                   & \\
1D & \textasciicircum ] & -& \\
1E & \textasciicircum \textasciicircum & begin Map data & Starts a block immediately followed by an index type identifier
(Integer / Long / String), the number of entries (0 .. n), then field terminator, then the with key / value pairs (serialized
form of Java maps)\\
1F & \textasciicircum \_ &  unused         & \\
\hline
\end{tabular}

\vspace{8mm}

\subsection{Message format railroad diagrams}
In the following diagrams, a single uppercase letter inside a circle represents the single byte control code represented by the ASCII code of the letter minus 64 (0x40).
 
\railoptions{+ac}   % \railoptions{-h}


\begin{rail}
null : 'N'
    ;
   
atomicfield : value 'F'
    ;
   
array :
    'B' count 'F' (member *) 'A'
    ;

member : null | atomicfield | array | object
    ;

revision : 'N' | (revisionstring 'F')
    ;
   
\end{rail}
        
\begin{rail}
versionedmemberlist :
    revision (member*)
    ;
\end{rail}
        
\begin{rail}
object :
    'S' objectname 'F' versionedmemberlist
          (('P' versionedmemberlist) *)
    'O'
    ;
\end{rail}
        
\begin{rail}
record :
    'R' revision
                          (('C' comment) | (object))
                                                    ('M')? 'J'
    ;
\end{rail}
        
\begin{rail}
transmission :
    'T' revision
    (('X' extension 'Y')+) 
    ( record *) 
    ('Z')?
    ('Q')?
    ;
\end{rail}

\vspace{8mm}


\section{Message compatibility rules}
In a large distributed platform, extensions of data objects frequently occur. Atbon is designed to support upgrades of platform
components at different points in time. In order to support this, the
``cooperative extension'' model is used. This means, that some data model changes are supported, while others are not.

The changes not supported (while keeping the same version number) are:
\begin{itemize}
    \item Swapping of field order within a class. This is due to skipping the field names from the message for sake of
    compactness.
    \item Changing a data type (some changes are allowed, see exceptions below)
    \item Changing a class name
    \item adding required (not NULLable) fields
    \item Modifying the inheritance structure (i.e. switching between C extends A and C extends B, B extends A)
    \item Changing the index type of a map
\end{itemize}

The following changes are supported:
\begin{itemize}
    \item renaming a field within a class
    \item adding extra fields at the end of a class (even if this class is used as a base class and subclasses are serialized)
    \item changing a boolean into a number or character field
    \item changing an ASCII field into a UNICODE field
    \item removing an existing field, which is not the last field of a class.
    \item Swapping between array, List and Set data structures 
\end{itemize}

If incompatible changes cannot be avoided, use of the ``versioning feature'' is required. This requires additional development
at the application side.







\section{Backend implementations}

\subsubsection{Domain / package name mapping}
In order to convert between domain names and Java / Scala packages, the following rules are implemented, in order of increasing
priority:
\begin{itemize}
    \item Use of {\ttfamily de.jpaw.bonaparte.pojos.}$Domainname$ as a package name (fallback)
    \item Use of a customized $packageNamePart1$.$Domainname$.$packageNamePart2$ package name (configurable by static setter
    function).
    \item Use of a Java collections {\ttfamily Map<String,String>}, which maps between Domain and package, using the previous rules as a
    fallback, if the domain has not been found in the map.
\end{itemize}


\subsection{Interoperability / Message transport}
The core Bonaparte project focusses on POJO serialization only, but a number of implementations are provided as separate add-on projects, which offer seamless support for different transport libraries:

\subsubsection{Netty4}
Bonaparte integrates well with Netty4.
The bonaparte-netty project offers an API to transmit objects via Netty4.
This has been tested with Netty 4.0.18

\subsubsection{vert.x}
This has been tested with vert.x 2.0.1.Final

\subsubsection{HornetQ (JMS API)}
This has been tested with HornetQ 2.2.21 GA.

\subsubsection{Apache activeMQ Apollo (STOMP API)}
This has been tested only with an external Apollo server.

\subsubsection{Apache Camel}
This has been tested with Camel 2.12. There is an unexplained performance issue that Camel using the Bonaparte serialization is slower than Camel with GSON or Jackson,
 despite the serializer / deserializer of Bonaparte standalone is 2 to 5 times faster than GSON or Jackson. (It looks like the integration is not yet optimal.)

\subsubsection{Akka}
Akka is a very promising actor framework. Actors can communicate via the Bonaparte serialization format as shown in the bonaparte-akka sample project.

\section{License}
The Bonaparte DSL and the default serialization / deserialization implementation for Java
are available under the Apache license, version 2 (\url{http://www.apache.org/licenses/}). 

\section{References and Prerequisites}
The source of the Bonaparte DSL can be obtained here: \url{https://github.com/jpaw/bonaparte-dsl}.

At \url{https://github.com/jpaw/bonaparte-java}, there is a Java implementation of the (de)serializers for the Bonaparte message format. This repository also holds the transport integration projects.

All implementations require Java Standard Edition, version~7 ($1.7.0\_09$ or better), the DSL is based on Eclipse 4.2.2 (Juno
SR2) with XText / Xtend 2.4.1 or better (for Bonaparte-1.5.1 or later).
\end{document}
